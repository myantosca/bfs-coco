\documentclass[11pt,epsf]{article}
\usepackage{amssymb,amsmath,amsthm,amsfonts,mathrsfs,color}
\usepackage{epsfig}
\usepackage{latexsym}
\usepackage{verbatim}
\usepackage{setspace}
\usepackage{algorithm}
\usepackage[noend]{algorithmic}
\usepackage{algorithmicext}
\usepackage{ifthen}
\usepackage{graphicx}
\usepackage{url}
\usepackage{hyperref}
\usepackage[utf8]{luainputenc}
\usepackage[bibencoding=utf8,backend=biber]{biblatex}
\addbibresource{cosc6326-pa2-michael-yantosca.bib}
\usepackage{fancyhdr}
\pagestyle{fancy}
\lhead{{\footnotesize{COSC6326 PA 2}}}
\rhead{{\footnotesize{Michael Yantosca}}}

\usepackage{longtable}
\usepackage{pgfplots}
\usepackage{pgfplotstable}
\usepgfplotslibrary{external}
\usepgfplotslibrary{statistics}
\usepgfplotslibrary{groupplots}
\usetikzlibrary{pgfplots.groupplots, external}
\tikzexternalize[]
\pgfplotsset{
  tick label style={font=\footnotesize},
  label style={font=\small},
  legend style={font=\small},
  compat=newest
}
\pgfplotstableset{
  col sep=comma,
  begin table=\begin{longtable},
  end table=\end{longtable},
  every head row/.append style={after row=\endhead}
}

\newtheorem{fact}{Fact}
\newtheorem{theorem}{Theorem}
\newtheorem{lemma}{Lemma}
\newtheorem{claim}{Claim}
\newtheorem{remark}{Remark}
\newtheorem{definition}{Definition}
\newtheorem{corollary}{Corollary}
\newtheorem{proposition}{Proposition}
\newtheorem{example}{Example}
\newtheorem{observation}{Observation}
\newtheorem{exercise}{Exercise}
\newtheorem{statement}{Statement}
\newtheorem{problem}{Problem}

\newcommand{\TODO}[0]{\textbf{\color{red}{TODO}}}

% \linregplots{title}{prefix}{suffix}{x}{y}
\newcommand{\linregplots}[5]{
  \nextgroupplot[title=#1]
  \addplot [red, only marks, mark size=0.5] table [x=#4, y=#5] {#2.k1#3.log};
  \addplot [red, no markers] table [x=#4,y={create col/linear regression={y=#5}}] {#2.k1#3.log};
  \addplot [blue, only marks, mark size=0.5] table [x=#4, y=#5] {#2.k2#3.log};
  \addplot [blue, no markers] table [x=#4,y={create col/linear regression={y=#5}}] {#2.k2#3.log};
  \addplot [green, only marks, mark size=0.5] table [x=#4, y=#5] {#2.k4#3.log};
  \addplot [green, no markers] table [x=#4,y={create col/linear regression={y=#5}}] {#2.k4#3.log};
  \addplot [orange, only marks, mark size=0.5] table [x=#4, y=#5] {#2.k8#3.log};
  \addplot [orange, no markers] table [x=#4,y={create col/linear regression={y=#5}}] {#2.k8#3.log};
  \addplot [purple, only marks, mark size=0.5] table [x=#4, y=#5] {#2.k16#3.log};
  \addplot [purple, no markers] table [x=#4,y={create col/linear regression={y=#5}}] {#2.k16#3.log};
  \addplot [brown, only marks, mark size=0.5] table [x=#4, y=#5] {#2.k32#3.log};
  \addplot [brown, no markers] table [x=#4,y={create col/linear regression={y=#5}}] {#2.k32#3.log};
}

\date{}
\title{COSC6326 Programming Assignment 2}
\author{Michael Yantosca}
\begin{document}
\maketitle
\tableofcontents

\section{Introduction}{
  \paragraph{}{
    \TODO
  }
}

\section{Analysis}{
  \subsection{\texttt{text2mpig}}{
    \paragraph{}{
      The graph representations available from the SNAP database\autocite{SNAP} are in a textual format
      over which is difficult to efficiently parallelize data access. Whereas one could serially access
      the data from a predetermined root node and thereafter distribute it to other participating nodes,
      the loading time might be reduced if all the nodes could participate in the loading by each focusing
      on a subset of the data to be loaded. At rest, the vertex-centric model has a disadvantage against
      the edge-centric model since the variability of node composition in the absence of a guarantee of
      isomorphism or regularity necessitates additional header information on top of the unevenness of
      distribution. The optimal distribution of data cannot be determined \emph{a priori} by an unbiased
      strategy.
    }
    \paragraph{}{
      Conversely, the read load of an edge-centric model can be evenly distributed when
      encoded in an efficient packed binary format since all edges are symmetric with each
      other provided that the vertex labels can be expressed in a fixed-width format. It is
      likely for this reason that the SNAP database provides models as a set of edges. A
      little preprocessing work up front could yield dividends, even when the algorithm
      model is vertex-centric. To this end, the \texttt{txt2mpig} utility was created with
      the additional benefit of a decrease in storage costs at rest since each node label
      only requires 4 bytes (each edge 8 bytes).
    }
    \paragraph{}{
      \begin{algorithm}
        \footnotesize
        \caption{\textsc{Text-To-MPI-Graph}, Algorithm for Converting Edge-Centric Text Model to Compact Edge-Centric Form}
        \begin{algorithmic}
          \REQUIRE{$e_B$, the number of edges per output block per machine}
          \REQUIRE{$k$, the number of participating nodes}
          \REQUIRE{$r$, the participating process rank}
          \STATE{Open the output file for writing each edge $(u,v)$ as <$u$|$v$>.}
          \WHILE{\neg \textsc{eof}}{
            \IF{$r = 0$}{
              \STATE{Read a line of text.}
              \IF{the line is of the form ``$u$ $v$''}{
                \STATE{Add <$u$|$v$> to the binary read buffer.}
              }\ELSE{
                \STATE{Discard the line.}
              }\ENDIF
              \STATE{Check for \textsc{eof}.}
            }\ENDIF
            \STATE{Broadcast the updated read offset to all processes.}
            \STATE{Broadcast the \textsc{eof} status to all processes.}
            \IF{read buffer is at capacity ($16e_Bk$ bytes) or \textsc{eof}}{
              \STATE{Wait for the previous non-blocking write to finish.}
              \STATE{Scatter the read buffer evenly among all the processes.}
              \STATE{Start a non-blocking write at an offset linearly scaled by $r$ to avoid contention.}
              \STATE{Increment the absolute write offset for the next pass.}
              \STATE{Tare the read buffer offset.}
            }\ENDIF
          }\ENDWHILE
          \STATE{Wait for any outstanding disk writes.}
          \STATE{Close the output file.}
        \end{algorithmic}
      \end{algorithm}
    }
    \paragraph{}{
      The resulting edge-centric binary files exhibited expected compression by virtue
      of being more efficient on average in terms of label storage.
      \begin{longtable}{l|c|c}
        \textbf{Real-World Graph} & \textbf{Text File Size} & \textbf{Binary File Size} \\
        facebook\_combined & 835K & 690K \\
        ca-AstroPh & 5.1M & 3.1M \\
        roadNet-TX & 57M & 30M \\
      \end{longtable}
    }
  }

  \subsection{\texttt{genmpig}}{
    \paragraph{}{
      The same binary file format could easily be generated in a stream-processing fashion
      for an Erdos-Renyi probabilistic graph. Two variants of the algorithm were created,
      largely out of a necessity to find a more time-efficient graph generator.
      Unfortunately, as the names suggest, the latter algorithm sacrificed some
      correctness for the sake of speed.
    }
    \paragraph{}{
      \begin{algorithm}
        \footnotesize
        \caption{\textsc{Generate-MPI-Graph}, Distributed Algorithm for Generating an Erdos-Renyi Graph Stored in Compact Edge-Centric Form}
        \begin{algorithmic}
          \REQUIRE{$e_B$, edges per output block per machine}
          \REQUIRE{$p$, the probability of any given edge}
          \REQUIRE{$n$, the graph population size (nodes)}
          \REQUIRE{$r$, the process rank}
          \STATE{Seed a PRNG for a Bernoulli distribution with decision probability $p$.}
          \STATE{Open the output file for writing each edge $(u,v)$ as <$u$|$v$>.}
          \FOR{$i \in [0,n-1]$}{
            \STATE{$u \gets i + r$}
            \IF{$u < n$}{
              \STATE{Add an ``edge'' $(u,u)$ to ensure inclusion of $u$ even as a singleton.}
              \IF{write buffer has reached capacity}{
                \STATE{Dump the buffer via a shared file pointer.}
                \STATE{Tare the write buffer offset.}
              }\ENDIF
              \STATE{Generates incident edges from $u$ according to Algorithm~\ref{alg:chooselv} or Algorithm~\ref{alg:choosemc}}
            }\ENDIF
          }\ENDFOR
          \STATE{Dump remainder of write buffer if partially full.}
          \STATE{Wait for final write.}
          \STATE{Close the output file.}
        \end{algorithmic}
      \end{algorithm}
    }
    \paragraph{}{
      \begin{algorithm}
        \footnotesize
        \caption{\textsc{Choose-Edges-Las-Vegas}, Naive Edge Selection Algorithm with Guaranteed Correctness}
        \label{alg:chooselv}
        \begin{algorithmic}
          \REQUIRE{$u$, the source node label}
          \REQUIRE{$P_{E}$, a Bernoulli distribution with decision probability $p$}
          \FOR{$v \in [u + 1, n - 1]$}{
            \STATE{Take sample $E_{uv}$ from space with distribution $P_{E}$.}
            \IF{$E_{uv} = 1$}{
              \STATE{Add <$u$|$v$> to the binary write buffer.}
              \IF{write buffer has reached capacity}{
                \STATE{Dump the buffer via a shared file pointer.}
                \STATE{Tare the write buffer offset.}
              }\ENDIF
            }\ENDIF
          }\ENDFOR
        \end{algorithmic}
      \end{algorithm}
    }
    \paragraph{}{
      \begin{algorithm}
        \footnotesize
        \caption{\textsc{Choose-Edges-Monte-Carlo}, Time-Efficient Edge Selection Algorithm with Suboptimal Correctness}
        \label{alg:choosemc}
        \begin{algorithmic}
          \REQUIRE{$u$, the source node label}
          \REQUIRE{$P_v$, a uniform distribution over $[u+1, n-1]$}
          \REQUIRE{$B_E$, a $Bin(n - 1 - u, p)$ distribution}
          \STATE{Sample $B_E$ to get a number of successful edge creations $E_u$.}
          \FOR{$e \in [0, E_u-1]$}{
            \STATE{$v \gets \text{ a sample from } P_v$}
            \STATE{Add <$u$|$v$> to the binary write buffer.}
            \IF{write buffer has reached capacity}{
              \STATE{Dump the buffer via a shared file pointer.}
              \STATE{Tare the write buffer offset.}
            }\ENDIF
          }\ENDFOR
        \end{algorithmic}
      \end{algorithm}
    }
    \paragraph{}{
      The asymptotic time benefits of Algorithm~\ref{alg:choosemc} over
      Algorithm~\ref{alg:chooselv} are fairly apparent, particularly when one considers
      the overhead necessary to generate a combinatorial set of coin flips.
      However, the use of the uniform distribution to choose an $O(np)$ sequence
      of numbers runs the risk of sampling duplicate edges. Even when $p = 1$,
      the algorithm did not always generate a complete graph. A simple shuffle
      algorithm would be a better choice if not for the onerous \emph{space}
      requirements at the data scales under consideration.
    }
  }

  \subsection{\texttt{bfs-coco}}{
    \paragraph{}{
      \begin{algorithm}
        \footnotesize
        \caption{\textsc{BFS-Connected-Components}, Distributed BFS Algorithm for Determining Connected Components in an Arbitrary Graph}
        \label{alg:bfs-coco}
        \begin{algorithmic}
          \STATE{\TODO}
        \end{algorithmic}
      \end{algorithm}
    }
    \paragraph{}{
      \TODO
    }

    \begin{theorem}
      \label{thm:bfs-coco}
      Algorithm~\ref{alg:bfs-coco} determines the connected components in an arbitrary graph
      of $n$ vertices distributed over $k$ machines with communication complexity $O(???)$.
    \end{theorem}
    \begin{proof}
      Because reasons.
    \end{proof}
  }
}

\section{Results}{
  \subsection{Test Procedures}{
    \paragraph{}{
      Tests to validate correctness were performed locally on a dual-core laptop
      \footnote{2 x Intel(R) Core(TM) i5-6200U CPU @ 2.30GHz.} running Pop!OS.
      \footnote{An Ubuntu 18.04 variant.}
      Tests for which results were collected systematically and graphed were
      done on the UH \texttt{crill} cluster with the following parameters:
      \begin{itemize}
      \item{$n \in 2^{[10,20]}$, the total population size}
      \item{$k \in \{1,2,4,8,16,32\}$, the number of distributed nodes}
      \item{$\epsilon = 0.2$, the threshold error}
      \item{$p$, the existential probability of a given edge,
        \begin{align}
          p &
          \begin{cases}
            < \frac{(1 - \epsilon)\ln n}{n} \\
            = \frac{\ln n}{n} \\
            > \frac{(1 + \epsilon)\ln n}{n} \\
            < \frac{(1 - \epsilon)}{n} \\
            = \frac{1}{n} \\
            > \frac{(1 + \epsilon)}{n} \\
          \end{cases}
        \end{align}
      }
      \end{itemize}
      Each parameter combination was executed at least 3 times with
      several executables contributing to the testing process:
    }
    \paragraph{\texttt{txt2mpig}}{
      A utility program for converting SNAP edge-centric model text files
      into a compact edge-centric binary format suitable for accessing via MP/IO.
    }
    \paragraph{\texttt{genmpig}}{
      A utility program for generating Erdos-Renyi random graphs and saving
      in a compact edge-centric binary format.
    }
    \paragraph{\texttt{bfs-coco}}{
      A program which reads a compact edge-centric binary format file and
      distributes the vertex-centric equivalent across a $k$-machine context
      and subsequently executes Algorithm~\ref{alg:bfs-coco} on the distributed graph.
    }
    \paragraph{}{
      The reader is directed to the accompanying \texttt{README.md} for
      explicit usage instructions on the various programs as well as
      for a deeper explanation of implementation considerations,
      engineering tradeoffs, and known issues with the programs.
    }
    \paragraph{}{
      It should be noted that the MPI 2.1 standard\autocite{MPI21} and g++-7.2
      was used for development since they were available through the development
      laptop's package system, but the program compiled and ran on the \texttt{crill}
      cluster with g++-5.3.0 and MPI 3.0. The code requires the C++-11 standard.
    }
  }
}

\section{Conclusions}{
  \paragraph{}{
    \TODO
  }
}

\printbibliography
\end{document}
